\documentclass[a4paper]{article}

\setcounter{tocdepth}{3}

% latex package inclusions
\usepackage{fullpage}
\usepackage{hyperref}
\usepackage{tabulary}
\usepackage{amsthm}
\usepackage{xcolor}

% set up BNF generator
\usepackage{syntax}
\setlength{\grammarparsep}{10pt plus 1pt minus 1pt}
\setlength{\grammarindent}{10em}
\setlength{\parindent}{0cm}


% set up source code inclusion
\usepackage{listings}
\lstset{
  tabsize=2,
  basicstyle = \ttfamily\small,
  columns=fullflexible
}

% in-line code styling
\newcommand{\shell}[1]{\lstinline{#1}}

\theoremstyle{definition}
\newtheorem{question}{Gap}


%%%%%%%%%%%%%%%%%%%%%%%%%%%%%%%%%%%%%%%%%%%%%%%%%%%

\begin{document}
\title{The WACC Compiler: Report}
\date{9th December 2016}
\author{
Group 10: Ben Sheng Tan, Adanna Akwataghibe, Mark Aduol, Alessio Incitti
}

\maketitle
%%%%%%%%%%%%%%%%%%%%%%%%%%%%%%%%%%%%%%%%%%%%%%%%%%%%%%%%%%%%%%%%%%%%%%%%
%%
\section{The Product}
%%%%%%%%%%%%%%%%%%%%%%%%%%%%%%%%%%%%%%%%%%%%%%%%%%%%%%%%%%%%%%%%%%%%%%%%%%
WACC is a simple programming language which supports function declarations, if-else conditionals, while loops, pairs and arrays.
The WACC compiler we developed using Java and ANTLR supports all basic functionalities and generates suitable assembly code for the ARM11 processor. The compiler is split into two processes: the front end, which consists of lexical, syntactic and semantic analysis of the WACC program; and the back end, which generates the assembly code required to run on an ARM11 processor.

\subsection{Front End Functionality}
%%%%%%%%%%%%%%%%%%%%%%%%%%%%%%%%%%%%
We decided to approach front end with ANTLR tools provided. It begins with a lexical analysis of the input WACC program, separating it into tokens based on the grammar specified in the lexer file. The WACCParser then proceeds to check for syntactic errors. If there are no errors, the compiler will develop the program's parse tree and proceed to semantic analysis for type checking of variables and expressions. Semantic errors will be thrown if there are errors like multiple declarations of the same variable, absence of return statements, mismatch of the return type and incompatible parameters in function calls are detected. The semantic checker will ensure that the type of the expression for conditionals and loops to be boolean. Also, the compiler should be able to check all types of expressions.

\subsection{Back End Functionality}
%%%%%%%%%%%%%%%%%%%%%%%%%%%%%%%%%%%
The implementation of backend is capable of generating accurate machine code using the parse tree from the Front End, that works on ARM11 architectures. The parse tree built in frontend was a crucial part in the ease of development, as it provides a sort of modular system in which the specific code blocks for each type of command can be added. Our unique approach at the back end has allowed us to produce machine code within the same visit as the semantic check.  As each node on the tree is visited in the back end, the node provides the block of assembly for each command. This allowed for minimal duplication and efficient implementation of the basic WACC language specifies. The resulting assembly code is written to a .s file and saved in the root directory. Most of the WACC functionalities are considered in our implementation, meeting the functional specification.

\subsection{Possible Further Development}
%%%%%%%%%%%%%%%%%%%%%%%%%%%%%%%%%%%
Some important aspects that we failed to implement completely were the assignments of array elements, as well as ASCI Table. This may provide us with some issues when looking to extend our language in future as they are fairly fundamental aspects of the WACC specification. If we have the time, we would further improve and extend our compiler so that it supports more functionality. We would also like to implement concurrency for WACC by writing a scheduler. Concurrency in programs greatly reduces the execution time. If we implement this, we can produce very efficient WACC programs. Nonetheless, we would like to do more optimisations the assembly code generated as well as the compiler.

% TOWRITE: issue you think we have with the current version and way to improve it for further development

\subsection{Performance Issue}
%%%%%%%%%%%%%%%%%%%%%%%%%%%%%%%%%%%
The way our compiler is built requires one visit of the generated syntax tree to accomplish syntax and semantic check,and produce ARM code reduces the execution time, being the main ‘edge’ of our compiler.

% TOWRITE: further elaboration linking to optimisation

%%%%%%%%%%%%%%%%%%%%%%%%%%%%%%%%%%%%%%%%%%%%%%%%%%%%%%%%%%%%%%%%%%%%%%%%%%
\section{The Project Management}
%%%%%%%%%%%%%%%%%%%%%%%%%%%%%%%%%%%%%%%%%%%%%%%%%%%%%%%%%%%%%%%%%%%%%%%%
\subsection{Group Organisation}
%%%%%%%%%%%%%%%%%%%%%%%%%%%%
We split up the work by identifying the major tasks indicated on the specification. For instance, four tasks are notable in the Front-End: Lexer, Parser, Syntax and Semantic checkers. We decided for two of us creating lexer and parser first before starting to work on Syntax/Semantic checkers together. We divided the parts into statements, expressions, functions and types and all of us works on our own package. The same is applied to backend. To organise our project and monitor the progress of each member, we have had regular meetings to discuss where we are up to. Most of us usually work in lab together because we can have direct communication with each other and discuss debug problems.

\subsection{Use of Project Management Tools}
%%%%%%%%%%%%%%%%%%%%%%%%%%%%
\subsubsection{Version control with Git}
%%%%%%%%%%
We used git version control to manage our code throughout the project. Most of the time three of us worked on master branch and another mate worked on his own branch. For three of us, we will notify each other in our chat when we decided to push or change our codes to prevent merge conflicts. However, when we were updating codes in a very fast rate, it is hard to prevent conflicts to appear.
\\
\phantom{3ex} We have had git merge problems for several times and it is very time consuming to fix them. There was a time when we had a git merge problem with the TypeCheckVisitor for the front-end implementation. We spent a lot of time figuring out which version of the codes was the correct one when we have to merge ours and other's codes together. After having these small problems with git, we were very careful of when to push our files and when to pull and merge. Moreover, we ensured that our commit messages were useful to help the other group members understanding the changes made. Also, our commit message includes user names so that we could see who made the changes on it.

\subsubsection{Communication through Facebook Messenger}
%%%%%%%%%%%
We decided to start a messenger group chat to instantly update our progress; to keep track of what everyone is doing and prevent git merge conflicts. Facebook chat is very convenient as it is available on mobile devices. Even if we are away from our computer, we still can get instant updates on the project progress. However, some of the groupmates are not used to checking the messages frequently causing some delays in our discussion.

\subsubsection{IntelliJ IDEA -- Java integrated development environment (IDE)}
%%%%%%%%%%%
IntelliJ IDEA is great tool that helped us massively in our project. Using an IDE helped us to spot errors more quickly as they would be underlined in red and we would not have to wait till compile time to find them. It also helped us to navigate around the project more easily, and to see the structure of our packages more clearly.

\subsection{Reflection on Our Project Management}
%%%%%%%%%%%%%%%%%%%%%%%%%%%%
The area where we struggled the most and that we tried to overcome throughout the project was the readability of the code produced.We sometimes managed to get stuck on a part of the project because we could not fix the problems of the code produced by one of our teammates, due to lack of comprehension. However, the worst part is when we tried to merge the work from sub-branch with the master branch. The structure of the project and the coding style are totally different, which means two different versions of project are produced. Our progress was delayed because the similar works were done on two different versions. Nevertheless, we tried to avoid the similar situation happening by sticking to one version we works on. The additional version will be classified as an additional package in the project to avoid any loss of the work. Moreover, what we could do differently next time is to discuss the implementation methods before working on our part.

%%%%%%%%%%%%%%%%%%%%%%%%%%%%%%%%%%%%%%%%%%%%%%%%%%%%%%%%%%%%%%%%%%%%%%%%%%
\section{The Design Choices}
%%%%%%%%%%%%%%%%%%%%%%%%%%%%%%%%%%%%%%%%%%%%%%%%%%%%%%%%%%%%%%%%%%%%%%%%%%
\subsection{Front End Design}
%%%%%%%%%%%%%%%%%%%%%%
Our front end makes use of visitor pattern that many different classes representing different nodes of a parse tree, and that we require our program to perform a different action on each of the nodes. For the error checking, an exception package is created to handle all the semantic and syntactic exception. The core designs are:

\begin{itemize}
\item\textbf{\textit{Implementation of Symbol Table Data Structure:}}
\\
By making use of symbol table to record the semantic information of variables, functions and types, these identifiers are bound to meanings in the symbol table. When occurrences of pre-defined identifiers are found, they are accessed by looking up the variable in the corresponding symbol table. Two classes are implemented which deals with creating a symbol table which contains the nodes creating in the frontend. The SymbolTable class is initialised to contain all the WACC types defined in the specification. By making uses of nested symbol tables in Expectation class, it allows us to accommodate for the concept of scope within the program, where each new level of symbol table represents a new scope caused by begins, functions and branches.
\end{itemize}

\begin{itemize}
\item\textbf{\textit{ Visitor Pattern used in TypeCheckVisitor class:}}
\\
We  employed a visitor pattern to traverse the parse tree that had been created in TypeCheckVisitor class. The visitor class proceeds to add to the symbol table by starting with the function declarations and then going through the WACC program’s main. In addition, with this design pattern, the program does not need to know in advance the class of node it is visiting. Using the visitor pattern we have avoided hard coding the different actions we want to perform on each class into each node class itself. This makes our parse tree more reusable, as we could have a different visitor visiting the same parse tree and performing, on each node, sets of actions different from the previous visitor. We have taken advantage of this characteristic when we implemented our extension. In the source code, there are multiple visitors, each visiting the same parse tree, however performing different actions as they visit each node class.
\end{itemize}

\begin{itemize}
\item\textbf{\textit{Abstract Tree class with class inheritance to build an useful Abstract Syntax Tree:}}
\\
In order to implement the semantic analysis, we chose to create two abstract functions within the tree class:
\\
\texttt{\noindent
\phantom{3ex}(a) public abstract boolean check( SymbolTable st, ParserRuleContext ctx );
\newline\noindent
\phantom{3ex}(b) public abstract BaseType getType();
}
\\
This meant that both functions were implemented in all the nodes of our tree, which extended tree class. The first function takes a Symbol Table as first parameter and the context from the parser tree as the second parameter, and compares both to give a boolean result. For the latter function, it is used to get the type of the context and is useful in comparing items in the check function.
\end{itemize}

\begin{itemize}
\item\textbf{\textit{Exception package handling the syntactic and semantic errors:}}
\\
We chose to handle syntax and semantic errors by removing the error listener that ANTLR had, and replacing it with our own error listener, called ErrorListener. We created an Exception class, which our SyntaxError class and SemanticError class extended. We then implemented our error listener so that when the parser found a syntax error, it would throw SyntaxError and abort with the exit code 100; similarly, SemanticError will be thrown with exit code 200 if the parser found a semantic error.
\end{itemize}

%%%%%%%%%%%%%%%%%%%%%%
\subsection{Back End Design}
In our backend, we extended the abstract tree class by including an additional abstract function:
\texttt{
\noindent
\phantom{3ex}(a) public abstract TokSeq assemblyCodeGenerating(Register register);
}
\\The same visitor class is used to generate assembly code. Two additional packages are declared in backend: System package deals with ARM code like when reading an int, we print p_read_int assembly. and tokens packages which deals with ARM instructions e.g load, move etc. The rest of the classes are for creating labels, registers and dealing with stack allocation.The core designs for backend are:
\begin{itemize}
\item\textbf{\textit{Abstract Token class with class inheritance to build all useful ARM commands:}}
\\
The abstract class Token contains methods to append and get string of the ARM instructions. Also has sub-packages which with implicit instructions like freeing an array or checking overflow or types of system errors. We also decided to have an TokenSeq class which implements Iterable of Token, for us to be able to have a way to add multiple ARM instructions.  In the tokens package, all of the methods extend the Token class, and by default, implement method that build the ARM instruction.
\end{itemize}

\begin{itemize}
\item\textbf{\textit{CodeGenerator class and extra functions within Main class takes a list of commands and writes them in the right assembly file:}}
\\
createAssemblyFile and parseFlags are implemented in main class to take the file from input in the command line and sets the flags returning command line that contains all the flag values and creates the buffered writer to write to the output file.
\end{itemize}

\begin{itemize}
\item\textbf{\textit{ Factory pattern to build the internal representation of the program:}}
\\
Having an idea of what we needed to produce (a list of commands), we decided to build an internal representation of the program (a tree). All we had to do was to call on the main Program node a method generating the commands for this program. The program itself was creating some commands and then was asking its children to generate their own commands (without knowing what these children were). The key point of the structure was that each node was generating its own commands and was only asked by their parent to generate these commands. This recursive structure therefore allowed each node to create its own commands independently.
\end{itemize}


%%%%%%%%%%%%%%%%%%%%%%%%%%%%%%%%%%%%%%%%%%%%%%%%%%%%%%%%%%%%%%%%%%%%%%%%%%
\section{Beyond the Specification}
%%%%%%%%%%%%%%%%%%%%%%%%%%%%%%%%%%%%%%%%%%%%%%%%%%%%%%%%%%%%%%%%%%%%%%%%%%
After completing the basic compiler, we decided to extend the functionality of the compiler and WACC programs so that it is equipped to do more powerful computations. Tests were written to check extensions.

\subsection{For Loops and Do-While Loops}
%%%%%%%%%%%%%%%%%%%%%%%%%%%%%%%%%%%%%%%
We altered the parser to include the grammar for both loops. \textit{for} is followed by a statement and two conditional expressions, whilst the do-while is just a mirror of the normal while loop. In for-loop,  we had to check that each of the loops’ condition was the right type. For do-while, the first iteration of this loop runs regardless of the condition.It continues iterating if  the condition is satisfied. If we had more time, we would have also implemented the foreach loop.

\subsection{\textit{'break'} and \textit{'continue'} Statement}
%%%%%%%%%%%%%%%%%%%%%%%%%%%%%%%%%%%%%%%%%%%%%%%
These were implemented similarly to statement nodes, but we needed to consider the syntax within a loop by implementing break and continue checker in the general stat class. The \textit{'JMP'} token for both statements is added in the backend token package for generating assembly code. The compiler exits the loop if it detects \textit{'break'}; continues looping from the start when it sees \textit{'continue'}.

\subsection{Binary, Octal and  Hexadecimal Laterals}
%%%%%%%%%%%%%%%%%%%%%%%%%%%%%%%%%%%%%%%%%
We extended the type in basetype class as well as created the nodes extending expression node in the frontend package. The literals are declared in the systemFormatterHandler and SystemPrintHandler to generate the assembly code. We also extended typecheckVisitor by adding visit functions.

\subsection{Side-effecting Expression}
%%%%%%%%%%%%%%%%%%%%%%%%%%%%%%%%%%%%%
We added (++) and (--) for the for loops, as well as (+=) and (-=) in the parser. We checked that the parser would check that these assignments had to be a variable of type int and also the parser needed to know that the loop assignments were a unary operator, whereas the (+=, -=) were binary operators with the LHS being a variable with an int, and the RHS an int litter.

\subsection{Conditional Assignment}
%%%%%%%%%%%%%%%%%%%%%%%%%%%%%%%%%%%%%
We implemented the conditional assignment or ternary operator used in java, (cond ? stat : stat). The ternary “statement” is actually an expression, so can also be used as an assignRHS. During the type checking, we made sure that both statements used in the ternary statement were the same type.

\subsection{Advanced Types}
%%%%%%%%%%%%%%%%%%%%
We edited the parser to recognize declarations of lists and maps. We tried to make the WACC compiler have more java-like features, so we declare a list like
\\
\texttt{\noindent
\phantom{3ex}list<int> nums = new linkedList<int>(); or list<int> nums = new arrayList<int>();
\newline\noindent
}
They can also be nested lists and maps and lists of maps and vice versa. Similarly, we declare maps like:
\\
\texttt{\noindent
\phantom{3ex}map<char, int> lettersMap = new hashMap<char, int>();
\newline\noindent
}
We made the RHS have the type to make it easier for the compiler to rigorously test the types given.

\subsection{Optimisation : Instruction Evaluation}
%%%%%%%%%%%%%%%%%%%%%%%%%%%%%%%%%
We provided an optimiser that simplifies the generated ARM code before its output. The optimiser is implemented as static and final as we only use it once for each compilation. It takes as argument the generated ARM code, analyses it, removes redundant instructions and returns an optimised version. \\
\phantom{3ex} Code analysis is performed by the analyse methods, which goes through the ARM code using 2 StringTokenizers to separate instructions and its type from the operands. For each instruction, its type is stored as key and the operands as value in a hash map. The removal of redundancies is performed by removeRedundant which goes through the map entries using an iterator and compares every two subsequent instructions. We check if the two instructions are identical (by method removeDuplicates) and if they are same-effect instructions. If they are, the latter instruction is redundant and gets removed. This design allows easy insertion of additional checks by creating new methods that perform the check between the current and next instruction and by adding a method call inside removeRedundant’s loop. Thus, adding new checks will not severely impact performance, as the map is only walked once regardless of the number of checks. Finishing the process, the optimiser returns the code to the main class, which finally outputs it.

\subsection{Optimisation : Efficient Register Allocation}
%%%%%%%%%%%%%%%%%%%%%%%%%%%%%%%%%
We will use the word \textit{identifier}, rather than \textit{variable}, to refer to variables in our source program, to avoid confusion with "temporary variables" that will be introduced in the next sub-section. We say an identifier is \textit{live} at some point in our program if the value it contains at that point may be used in future computations. Conversely, it is \textit{dead} if there is no way its value can be used. A new version for backend is implemented for the support of efficient register allocation algorithm in our program, as follows:\\
\phantom{3ex} Firstly, we assume our machine has an infinite number of registers, and represent this behaviour by storing identifiers and values in newly created \textit{temporary variables}. We then say that an instruction \textit{generates liveness} for a particular variable if it is used as a source operand in that instruction, and \textit{destroys liveness} for the variable if it is used as a destination operand. Secondly, we calculate the the set of instructions that are reachable, in one iteration, from a particular instruction. We call this set the \textit{successor set} of the instruction. Now combining these pieces of info about the data-flow in our program, we are able to calculate which variables are live at which points in the program.\\
\phantom{3ex}
Variables that interfere with one another are live at the same point in the program so cannot share a register. We represent the control-flow of our program in a \texttt{ControlFlowGraph} class, and the interference of variables in a \texttt{InterferenceGraph} class. We then use a pair of mutually recursive methods \texttt{getLiveInSet} and \texttt{getLiveOutSet} that take as arguments the \texttt{ControlFlowGraph} and an integer index for an instruction, and calculate the set of variables that are live at the start and end of that instruction in the program. This is performed for all instructions in the program and the information gained is then used to build an \texttt{InterferenceGraph}. Thus the problem of finding a suitable register allocation is reduced to finding a colouring of this graph such that no adjacent nodes share the same colour. This step of the algorithm is then performed in the final part of our top-level \texttt{allocate} method.

%%%%%%%%%%%%%%%%%%%%%%%%%%%%%%%%%%%%%%%%%%%%%%%%%%%%%%%%%%%%%%%%%%%%%%%%%%%%%%%%%%%%%%%%%%%
\end{document}